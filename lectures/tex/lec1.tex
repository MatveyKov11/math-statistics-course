\documentclass[12pt,a4paper]{article}
\usepackage[utf8]{inputenc}
\usepackage[russian]{babel}
\usepackage[OT1]{fontenc}
\usepackage{amsmath}
\usepackage{amsfonts}
\usepackage{amssymb}
\usepackage{graphicx}
\usepackage{subfigure}
\topmargin=-0.54cm
\textheight=25.7cm
\oddsidemargin=-0.04cm
\textwidth=17cm
\newcommand{\pict}[1]{\includegraphics[scale=0.04]{#1}}
\newcommand{\bin}[2]{\left( \begin{matrix} #1 \\ #2 \end{matrix} \right)}
\newcommand{\spl}[2]{\{ \begin{matrix} #1 \\ #2 \end{matrix} \}}
\begin{document}
Выборкой из генеральной совокупности объёма $n$ с распределением $P$ называется последовательность $X_1, \ldots, X_n$ н. о. р. с. в. (независимых одинаково распределённых случайных величин) с распределением $P$ \par 
Обозначение: $Low(X_i) = P$, или $X_i \sim P$. \par 
$X = (X_1, \ldots, X_n)^T$ \par 
$x_i = X_i(\omega)$, $\omega$ (элементарное событие) одно и то же при всех $i$, мы определяем $\Omega$ (пространство элементарных событий) с помощью проведения опыта \par 
$x = (x_1, \ldots, x_n)^T$ - всё, с чем имеет дело матстатистика, $x$ тоже называют выборкой, но после опыта \par 
Упорядоченная выборка называется вариационным рядом $X_{(1)} \leq X_{(2)} \leq \ldots \leq X_{(n)}$, члены этого ряда называются порядковыми статистиками: $X_{(k)}$ - $k$-ая порядковая статистика \par 
Если $\omega$ фиксированно, то на данной выборке можно построить распределение, оно и его функция распределения будут обычными. Если $\omega$ не зафиксированно, то распределение и функция распределения будут случайными \par 
$F_n(x) = \frac{1}{n}\sum\limits_{i=1}^{n}\chi(x - X_i) = \frac{K}{n}$ - оценка распределения, где $K$ - число элементов выборки, меньших $x$, $\chi$ - ступенька Хевисайда, $\chi(x) = \left\lbrace \begin{matrix}
1, & x > 0 \\
0, & x \leq 0
\end{matrix}\right.$ \par 
$F_n(x)$ - эмпирическая функция распределения, при котором мы подразумеваем, что вероятность получения каждого значения $X_i$ одна и та же \par 
$EF_n(x) = \frac{1}{n}\sum\limits_{i=1}^{n}E\chi(x - X_i) = \frac{1}{n}\sum\limits_{i=1}^{n}\mathbb{P}(\{X_i < x\}) = \frac{1}{n}\sum\limits_{i=1}^{n}F(x) = F(x)$ - матожидание функции равномерного распределения совпадает с функцией распределения генеральной совокупности \par 
$\mathbb{P}(\{ X_{(k)} \in [x,\ x + dx)\} ) = $ * $k-1$ элемент $< x$, один элемент в интервале (всего элементов $n$) * $= n\mathbb{P}(\{ X_1 \in [x,\ x+dx); \text{ ровно } k-1 \text{ элемент выборки } < x\} ) = nP([x,\ x+dx))C_{n-1}^{k-1}F(x)^{k-1}(1 - F(x))^{n-k} + o(dx)$, устремим $dx \rightarrow 0$ и получим плотность $k$-ой порядковой статистики \par
$f_{(k)} = nf(x)C_{n-1}^{k-1}F(x)^{k-1}(1 - F(x))^{n-k}$ \par
Свойства $F_n(x)$ - эмпирической функции распределения: \par
1)$EF_n(x) = F(x)\ \forall x \in \mathbb{R}^1$ \par 
Свойство 1) - свойство несмещённости оценки, $F_n(x)$ в среднем совпадает с $F(x)$ \par 
2) По ЗБЧ (закону больших чисел) $F_n(x) \rightarrow^{\mathbb{P}} F(x)$ т. е. $\forall \epsilon > 0\ \mathbb{P}(\{|F_n(x) - F(x)| > \epsilon\}) \rightarrow_{n \rightarrow \infty} 0$ \par 
Свойство 2) - свойство состоятельности оценки, $F_n(x)$ близко к $F(x)$ в каком-то смысле \par 
3)$\sqrt{n}(F_n(x)-F(x)) = \frac{1}{\sqrt{n}}\sum\limits_{i=1}^{n}(\chi(x - X_i)-F(x)) \rightarrow^d N(0, \sigma^2)$ по центральной предельной теореме, где $\sigma^2 = E\chi^2(x - X_i) - (F(x))^2 = F(x)(1 - F(x))$ \par
Свойство 3) - свойство асимптотической нормальности, позволяет оценивать погрешность \par 
4) Теорема Гливенко \par 
$\sup\limits_{x\in\mathbb{R}^1}|F_n(x) - F(x)| \rightarrow_{n \rightarrow \infty} 0$ с вероятностью 1 \par 
5) Теорема Колмогорова \par 
$\sqrt{n}\sup\limits_{x\in\mathbb{R}^1}|F_n(x) - F(x)| \rightarrow^{d} \varsigma \sim K$ \par
$\varsigma$ имеет распределение Колмогорова, $\mathbb{P}(\varsigma < t) = \sum\limits_{j=-\infty}^{+\infty}(-1)^i e^{-2j^2t^2}$ \par 
Метод подстановки для получения оценок функционалов от функции распределения генеральной совокупности \par 
Идея в замене в функционале функции $F$ на функцию $F_n$ \par 
Оценивание моментов генеральной совокупности \par 
\end{document}