\documentclass[12pt,a4paper]{article}
\usepackage[utf8]{inputenc}
\usepackage[russian]{babel}
\usepackage[OT1]{fontenc}
\usepackage{amsmath}
\usepackage{amsfonts}
\usepackage{amssymb}
\usepackage{graphicx}
\usepackage{subfigure}
\topmargin=-0.54cm
\textheight=25.7cm
\oddsidemargin=-0.04cm
\textwidth=17cm
\newcommand{\pict}[1]{\includegraphics[scale=0.04]{#1}}
\newcommand{\bin}[2]{\left( \begin{matrix} #1 \\ #2 \end{matrix} \right)}
\newcommand{\spl}[2]{\{ \begin{matrix} #1 \\ #2 \end{matrix} \}}
\begin{document}
Экспоненциальные семейства распределения и эффективные оценки для них \par 
$\{F(x, \theta);\ \theta \in \Theta\}$ \par
$f(x, \theta)$ - плотность \par 
Если семейство регулярно и $f(x. \theta) = \exp(A(\theta)B(x) + C(\theta) + D(x))$, то его называют экспоненциальным \par 
$U(X, \theta) = \frac{\partial}{\partial \theta}\ln L(X, \theta) = \frac{\partial}{\partial \theta} \sum\limits_{i = 1}^n(A(\theta)B(X_i) + C(\theta) + D(X_i)) = A'(\theta)\sum\limits_{i=1}^nB(X_i) + nC'(\theta) = nA'(\theta)(T^* - (-\frac{C'(\theta)}{A'(\theta)})) = U(X, \theta)$ \par
$T^* = \frac{1}{n}\sum\limits_{i = 1}^nB(X_i)$, $\tau(\theta) = - \frac{C'(\theta)}{A'(\theta)}$, $a(\theta) = \frac{1}{nA'(\theta)}$ \par 
Получаем условие Рао-Крамера \par 
$T^*-\tau(\theta) = a(\theta)U(X, \theta)$ \par 
Следовательно, $T^*$ является эффективной оценкой для $\tau(\theta) = - \frac{C'(\theta)}{A'(\theta)}$ \par 
Найдём дисперсию этой оценки \par 
$\tau(\theta) = \int\limits_{\mathbb{R}^n}T^*(x)L_n(x, \theta)\lambda(dx)$ \par
$\tau'(\theta) = \int\limits_{\mathbb{R}^n}T^*(x)U(x, \theta)L_n(x, \theta)\lambda(dx)$ \par
$\tau'(\theta) = E_{\theta}(T^*U(X, \theta)) = cov(T^*, U(X, \theta)) = cov(T^*, \frac{1}{a}(T^* - \tau(\theta))) = \frac{D_{\theta}T^*}{a}$ \par 
Отсюда $D_{\theta}T^* = a(\theta)\tau'(\theta) = \frac{1}{nA'(\theta)}\cdot(-\frac{C'(\theta)}{A'(\theta)})'$ - формула для дисперсии \par 
Пример 1 - $N(\alpha, \theta^2)$ \par 
$f(x, \theta) = \frac{1}{\sqrt{2\pi}\theta}\exp(-\frac{1}{2\theta^2}(x - \alpha)^2) = \exp(-\frac{1}{2\theta^2}(x - \alpha)^2 - \ln\theta - \ln\sqrt{2\pi})$ \par 
$A(\theta) = -\frac{1}{2\theta^2}$, $B(x) = (x - \alpha)^2$, $C(\theta) = -\ln\theta$, $D(x) = -\ln\sqrt{2\pi}$ \par 
$\tau(\theta) = - \frac{(-\ln\theta)'}{(-\frac{1}{2\theta^2})'} = \frac{\frac{1}{\theta}}{\frac{1}{\theta^3}} = \theta^2$ \par 
Эффективное оценивание возможно для $\theta^2$ \par 
$T^*(X) = \frac{1}{n}\sum\limits_{i = 1}^n(X_i - \alpha)^2$ \par 
$D_{\theta}T^* = a(\theta)\tau'(\theta) = \frac{1}{n\frac{1}{\theta^3}}2\theta = \frac{2}{n}\theta^4$ - наилучшее значение \par 
Пример 2 - $Bi(k, \theta)$ - $x$ - число успехов среди $k$ испытаний \par 
$f(x, \theta) = C_k^x\theta^x(1 - \theta)^{k - x} = \exp(x\ln\theta + (k - x)\ln(1 - \theta) + \ln C_k^x) = \exp((\ln\theta - \ln(1 - \theta))x + k\ln(1 - \theta) + \ln C_k^x)$ \par 
$\tau(\theta) = - \frac{C'(\theta)}{A'(\theta)} = \frac{-\frac{k}{1 - \theta}}{\frac{1}{\theta} + \frac{1}{1 - \theta}} = k\theta$ \par 
$T^* = \frac{1}{n}\sum\limits_{i = 1}^nX_i = \overline{X}$ \par 
$D_{\theta}T^* = \frac{1}{n}DX_1 = \frac{k\theta(1 - \theta)}{n}$ \par 
Нужна эффективная оценка для $\theta$ \par 
$\tilde{T} = \frac{1}{kn}\sum\limits_{i = 1}^nX_i$ \par 
$D_{\theta}\tilde{T} = \frac{1}{n}DX_1 = \frac{\theta(1 - \theta)}{kn}$ \par 
Для экспоненциальных семейств всегда можно найти эффективную оценку (и только для них) \par 
Теорема \par 
Если регулярная статистическая модель допускает эффективное оценивание хотя бы одной функции от параметра, то она является экспоненциальной моделью \par 
Доказательство \par 
Пусть $X_1$ - выборка объёма 1 \par 
И пусть $T^*(X_1)$ - эффективная оценка некоторой функции $\tau(\theta)$ \par 
Тогда для неё выполнено условие Рао-Крамера \par 
$T^*(X_1) - \tau(\theta) = a_1(\theta)U(X_1, \theta)$ ($U$ - вклад одного элемента выборки) \par 
$T^*(X_1) - \tau(\theta) = a_1(\theta)\frac{\frac{\partial}{\partial \theta}f(X_1, \theta)}{f(X_1, \theta)}$ \par 
Для почти всех $x$ верно следующее: \par 
$T^*(x) - \tau(\theta) = a_1(\theta)\frac{\frac{\partial}{\partial \theta}f(x, \theta)}{f(x, \theta)}$ \par
Делим на $a_1$, интегрируем по $\theta$ и получаем ($\ln D(x)$ вылезает после взятия интеграла): \par 
$\int\frac{1}{a_1(\theta)}(T^*(x) - \tau(\theta))d\theta = \ln f(x, \theta) - \ln D(x)$ \par 
$A(\theta) = \int\frac{d\theta}{a_1(\theta)}$, $B(x) = T^*(x)$, $C(\theta) = -\int\frac{\tau(\theta)}{a_1(\theta)}$ \par 
Потенциализируем и в итоге: \par 
$\exp(A(\theta)B(x) + C(\theta) + D(x)) = f(x, \theta)$ - QED \par 
Как получать оптимальную оценку для других распределений? \par 
Критерий оптимальности Бхаттачария \par 
$U = \frac{\partial}{\partial \theta}\ln L(X, \theta) = \frac{L'_{\theta}(X, \theta)}{L(X, \theta)}$ \par 
В критерии Бхаттачария кроме выписанной дроби используются ещё дроби вида \par 
$\frac{L_{\theta}^{(2)}(X, \theta)}{L(X, \theta)}, \ldots, \frac{L_{\theta}^{(n)}(X, \theta)}{L(X, \theta)}$ \par 
Ищется оценка, для которой $T - \tau(\theta)$ является линейной комбинацией указанных дробей с коэффицентами, зависящими от параметра \par 
Неравенство и условие Бхаттачария \par 
Пусть $T(X)$ - несмещённая оценка для функции от параметра $\theta$, $E_{\theta}T(X) = \tau(\theta)$ \par 
Тогда её дисперсия удовлетворяет условию $D_{\theta}T \geq \sum\limits_{i,j = 1}^rc_{i,j}a_ia_j$, где $c_{i,j} = E_{\theta}(\frac{L^{(i)}}{L}\cdot\frac{L^{(j)}}{L})$, где коэффиценты определяются системой уравнений \par
$\tau^{(i)}(\theta) = \sum\limits_{j = 1}^rc_{i, j}(\theta)a_j(\theta)$, $i = 1, \ldots, r$ \par 
Знак в равенстве неравенства Бхаттачария достигается тогда и только тогда, когда $T - \tau(\theta) = \sum\limits_{i = 1}^ra_i\frac{L^{(i)}}{L}$ \par 
Замечание \par 
Если матрица $(c_{i,j})_{i,j = 1}^r$ является невырожденной, то неравенство Бхаттачария примет вид $D_{\theta} \geq \sum\limits_{i,j = 1}^rc^{i,j}\tau^{(i)}\tau^{(j)}$, где $c^{i,j}$ - элементы обратной матрицы \par 
$a = c^{-1}\tau'$ \par 
$(\tau')^T(c^{-1})^Tcc^{-1}\tau = (\tau')^T(c^{-1})^T\tau = (\tau')^Tc^{-1}\tau$ \par 
Это обобщение неравенства Рао-Крамера \par 
\end{document}