\documentclass[12pt,a4paper]{article}
\usepackage[utf8]{inputenc}
\usepackage[russian]{babel}
\usepackage[OT1]{fontenc}
\usepackage{amsmath}
\usepackage{amsfonts}
\usepackage{amssymb}
\usepackage{graphicx}
\usepackage{subfigure}
\topmargin=-0.54cm
\textheight=25.7cm
\oddsidemargin=-0.04cm
\textwidth=17cm
\begin{document}
Теперь ищем статистики $T_1$ и $T_2$ \par 
$(\frac{X_{(n)}}{T})^n = g$ \par 
$\frac{X_{(n)}}{T} = \sqrt[n]{g}$ \par 
$T = \frac{X_{(n)}}{\sqrt[n]{g}}$ \par 
$T_1 = \frac{X_{(n)}}{\sqrt[n]{g_2}},\ T_2 = \frac{X_{(n)}}{\sqrt[n]{g_1}}$ \par 
Продолжение в следующей лекции
Длина интервала $l = X_{(n)}(\frac{1}{\sqrt[n]{g_1}} - \frac{1}{\sqrt[n]{g_2}}) = X_{(n)}(\frac{1}{\sqrt[n]{g_1}} - \frac{1}{\sqrt[n]{g_1 + \gamma}})$ \par 
$l' = X_{(n)}(-\frac{1}{n}\frac{1}{g_1^{1+1/n}} + \frac{1}{n}\frac{1}{(g_1+\gamma)^{1+1/n}}) \leq 0$, надо взять самое большое $g_1$ \par 
$g_1 = 1 - \gamma,\ g_2 = 1$ \par 
Оптимальный доверительный интервал - $(X_{(n)}; \frac{X_{(n)}}{\sqrt[n]{1 - \gamma}})$ \par 
Универсальная центральная статистика \par 
Если $F(x, \theta)$ - функция распределения выборки и она монотонна по $\theta$, то универсальную центральную статистику можно построить единственным способом: \par 
$G(X, \theta) = -\sum\limits_{i=1}^n\ln F(X_i, \theta)$ \par 
Проверим, что $G$ центральная \par 
1) $F,\ \ln,\ \sum$ - монотонны по параметру \par 
2) $\xi_i = -\ln F(X_i, \theta)$ \par 
$\mathbb{P}(\{\xi < x\}) = \mathbb{P}(\{-\ln F(X_i, \theta) < x\}) = \mathbb{P}(\{F(X_i, \theta) > e^{-x}\}) = \mathbb{P}(\{ X_i > F^{-1}(e^{-x})\}) = 1 - F(F^{-1}(e^{-x})) = 1 - e^{-x}$ \par 
Распределение $\xi$ экспоненциальное с плотностью \par 
$f(x) = \left\lbrace \begin{matrix}
	e^{-x}, & x > 0 \\
	0, & x \leq 0
\end{matrix} \right.$ \par 
$G$ как сумма экспоненциальных случайных величин имеет гамма распределение с плотностью \par 
$f(x) = \left\lbrace \begin{matrix}
	\frac{x^{n-1}}{\Gamma(n)}e^{-x}, & x > 0 \\
	0, & x \leq 0
\end{matrix} \right.$ \par 
Пример применения универсальной центральной статистики на $R((0, \theta))$ \par 
Выбираем $g_1$ и $g_2$ как квантили гамма распределения, $F_G(g_2) - F_G(g_1) = \gamma$ \par 
$F(x, \theta) = \left\lbrace \begin{matrix}
	\frac{x}{\theta}, & x \in [0; \theta] \\
	0, & x \notin [0; \theta]
\end{matrix} \right.$ \par 
$G(X, T) = -\sum\limits_{i=1}^n\ln\frac{X_i}{T} = g$ \par 
$-\sum\limits_{i=1}^n\ln X_i + n\ln T = g$ \par 
$\frac{T^n}{\prod\limits_{i=1}^nX_i} = e^g$ \par 
$T = e^{g/n}(\prod\limits_{i=1}^nX_i)^{1/n}$ \par 
$T_1 = e^{g_1/n}(\prod\limits_{i=1}^nX_i)^{1/n},\ T_2 = e^{g_2/n}(\prod\limits_{i=1}^nX_i)^{1/n}$ \par 
Доверительный интервал - $\left( e^{g_1/n}(\prod\limits_{i=1}^nX_i)^{1/n}; e^{g_2/n}(\prod\limits_{i=1}^nX_i)^{1/n} \right)$ \par 
Для центрального доверительного интервала в качестве $g_1$ берём квантиль порядка $\frac{1-\gamma}{2}$, а для $g_2$ берём квантиль порядка $\frac{1+\gamma}{2}$ \par 
Существуют и другие методы построения, но их не рассматриваем \par 
Асимптотические доверительные интервалы \par 
Пусть $\widehat{\theta}_n$ - оценка максимального правдоподобия для параметра $\theta$ \par 
$\sqrt{n}(\widehat{\theta}_n - \theta) \to^d \xi \sim N(0, \frac{1}{i_1(\theta)})$ \par 
$\sqrt{n}\sqrt{i_1(\theta)}(\widehat{\theta}_n - \theta) \to^d \xi \sim N(0, 1)$ \par 
Пусть $\gamma$ - уровень доверия, $c_{\gamma} = \Phi^{-1}(\frac{1+\gamma}{2}), -c_{\gamma} = \Phi^{-1}(\frac{1-\gamma}{2})$ \par 
$\mathbb{P}(\{\sqrt{n}\sqrt{i_1(\theta)}|\widehat{\theta}_n - \theta| < c_{\gamma}\}) \approx \Phi(\frac{1+\gamma}{2}) - \Phi(\frac{1-\gamma}{2}) = \gamma$ \par 
$|\widehat{\theta}_n - \theta| < \frac{c_{\gamma}}{\sqrt{n}\sqrt{i_1(\theta)}}$ с вероятностью, близкой к $\gamma$ \par 
Проблема - $i_1$ зависит от $\theta$. В силу состоятельности функций от оценки максимального правдоподобия заменим $i_1(\theta)$ на $i_1(\widehat{\theta}_i)$ \par 
Примерно так же доказывается, что $|\widehat{\theta}_n - \theta| < \frac{c_{\gamma}}{\sqrt{n}\sqrt{i_1(\widehat{\theta}_n)}}$ - асимптотический $\gamma$-доверительный интервал \par 
Пример - $Bi(1, \theta)$ \par 
$\overline{X}$ - оценка максимального правдоподобия, $i_1(\theta) = \frac{1}{\theta(1 - \theta)}$ \par 
Доверительный интервал - $\left( \overline{X} - c_{\gamma}\sqrt{\frac{\overline{X}(1 - \overline{X})}{n}}; \overline{X} + c_{\gamma}\sqrt{\frac{\overline{X}(1 - \overline{X})}{n}} \right)$
\end{document}